\documentclass[a4paper,11pt]{article}

%
% Do not change
\textheight = 220mm
\textwidth  = 150mm
\topmargin  = 10mm
\oddsidemargin  = 5.0mm
\evensidemargin = 5.0mm
\unitlength = 1mm


\usepackage[T1]{fontenc} % Use latin1 font encoding

\usepackage[swedish]{babel} % If in Swedish

\usepackage{csquotes} % For blockquote

\begin{document}

\let\rempage=\thepage
{
\renewcommand{\thepage}{\relax}
\begin{picture}(44,0)(15,10)%
\special{psfile=mahlogo-name.eps}
\end{picture}%

\vspace*{-30mm}
\hfill\begin{minipage}[t]{10em}\large
Teknik och samh�lle\\
Datavetenskap
\end{minipage}

\vspace*{45mm}
\begin{center}
{\bf\large
Examensarbete 

\small
15 h�gskolepo�ng, grundniv�
}

\vspace*{25mm}
\LARGE

The differences in requirement elicitation between community- and firm-driven open source software projects on Github.

\vspace*{8mm}
\large

Examensarbetets titel p� svenska %TODO

\vspace*{12mm}
\Large
%
% Author names
Teddy Andersson\\
Filip Harald

\vspace*{30mm}
\large
% picture TODO?
\end{center}

\vfill
\hspace*{-10mm}%
\begin{minipage}[t]{20em}
%
% Fill in correct data for you
Examen:~kandidatexamen 180~hp
\\
Huvudomr�de:~datavetenskap
\\
Program:~systemutvecklare
\\
Datum f�r slutseminarium:~2016-xx-xx %TODO
\end{minipage}
\hfill
\begin{minipage}[t]{20em}
Handledare: Nancy Russo
\\
Assisterande handledare: Aleksander Fabijan
\\
Examinator: Mr. X
\end{minipage}

\newpage

\mbox{}

\newpage

\section*{Sammanfattning}

Text p� svenska\ldots

\newpage

\mbox{}

\newpage

\section*{Abstract}

Text in English\ldots

\newpage

\mbox{}

\newpage
\tableofcontents
\newpage
\ifodd\value{page}\else\mbox{}\newpage\fi
\setcounter{page}{1}
\renewcommand{\thepage}{\rempage}

\section{Introduction}

\subsection{Background}
%introduce the reader to the section
Here we will introduce the reader to the introduction section...

	%Introduce the reader to OSS
In the area of software development systems must evolve, or they risk to lose market shares to competitors. However, maintaining such a system can be difficult, complicated, and time consuming. The tasks of adding new features, adding support for new hardware devices and platforms, system tuning, and defect fixing all become more difficult as a system ages and grows.\cite{M.W.Godfrey2000} Due to the difficulties of maintaining software systems\footnote{TODO: didn't it approach because of developers wanted code to be free?}, a new more open approach of development emerge during the late 90's\cite{Wikipedia2011Dec.142000}. Today this development process is referred to as Open-Source Software (OSS) Development. In the essay The Cathedral and the Bazaar by  Eric S. Raymond the progressive open-source development, is described in the following words.

\begin{displayquote}
	A great babbling bazaar of differing agendas and approaches out of which a coherent and stable system could seemingly emerge only by a succession of miracles.
\end{displayquote}

\cite{DiBonaChrisandOckman1999, Maccormack2006}, however provides a more understandable and serious description of the OSS development process 
\begin{displayquote}
OSS [\dots] is characterised by (a) the distribution of a program's source code (programming instructions) along with the binary version of the product and (b) a license that allows a user to make unlimited copies of, and modifications to, this product. \cite{DiBonaChrisandOckman1999, Maccormack2006}
\end{displayquote} 

	%Introduce the reader to Community-driven projects
Further more OSS is commonly developed by a community that cooperate via the Internet and never, or seldom, meet face to face. The number of developers can differ from a handful to thousands and are often geographically distributed. Developers voluntarily contributes to the software by implementing new features, fixing bugs and writing documentation \cite{Maccormack2006}. Well known community developed OSS systems have grown strong since the late 90's such as The Apache Server, RedHat or the success of Linux, a Unix-like operating system originally written by Linus Torvalds, but subsequently worked on by hundreds of other developers in the late 90's.\cite{M.W.Godfrey2000} The key decisions in project driven OSS communities are taken by a central group of software developers (core developers). Beyond this central group, are the peripheral developers these are developers who intermittently submit code contributions (called patches). The core group reviews the patches before they are incorporated into the projects source code. \cite{Crowston2006}

%Introduce the reader to Firm-driven development
The conventional methods have within software development emerged into what we call Agile methods, whom are based on a incremental development process. The Agile methods are popular to use in Software Companies for information systems development. Agile methods are claimed to encourage developers to be more flexible and efficient by means of arrangements in the development teams physical and social environment\cite{Jansson2015}. In recent years, however, we have seen a increased interest of using open source alongside with the Agile development\cite{Author2008} \cite{Jansson2015}. Big companies like Google have been striving towards a more open enviorment in both development projects but also in general work process\cite{How Google Works}\footnote{Fix a reference for the book in, couldn't find a pdf version, but i have it in the bookshelf in the apartment //Teddy}. Alongside with an interest and momentum of OSS it is now also considered to be, in a commercial settings, a more viable approach\cite{Author2008}. From a commercial stand point OSS development also promieses many advantages. Including reduced salary costs; reduced cycle time arising from' 'follow-the-sun'' software development; cross-site modularization of development work; access to a larger skilled developer pool; innovation and shared best practice; and closer proximity to customers\cite{Author2008}.\footnote{Include how open source with a firm in the background are structured}
	
%Explain shortly the differences in requirement elicitation between Agile and Open-source development

%New description approach

%The latter analogy points to the discussion involved in an Open-Source Software (OSS) development process, which takes use of the OSS development model. The composition of the essay was due to the success of Linux, a Unix-like operating system originally written by Linus Torvalds, but subsequently worked on by hundreds of other developers in the late 90's \cite{M.W.Godfrey2000}, since then the OSS development model has gained momentum and is now, in a commercial settings, seen as a viable approach\cite{Author2008}. The frist analogy that Raymond brings up points towards the conventional methods whom have evolved since the late 90's. In this evolution of project process, Agile methods have become popular for information systems development (IDS)\cite{Jansson2015}. Agile methods are claimed to encourage developers to be more flexible and efficient by means of arrangements in the development teams physical and social environment\cite{Jansson2015}. The encouragement from the Agile methods on teams can likewise have been a participating factor to the momentum of the OSS development. Looking  deeper into how the work is conducted in the OSS development its interesting to ask how, why and in what context agile method works. The momentum of OSS development may correlate to the raise of agile methods but with more economical benefits from a commercial settings standpoint. �gerfalk and Fitzgerald \cite{Author2008} coined the word opensourcing which the explains  have characterized offshore sourcing as "outsourcing to a global workforce. They also point to that opensourcing and OSS development in general promises many advantages. including reduced salary costs; reduced cycle time arising from' 'follow-the-sun'' software development; cross-site modularization of development work; access to a larger skilled developer pool; innovation and shared best practice; and closer proximity to customers. 

%Explained in \cite{Crowston2006} is that OSS communities, whether run by volunteers or by companies, have a similar structure. The structure have clear lines to what an agile team looks like. At the center of the structure is the core group this is the group of people who are the key decision makers for the project. Beyond this core group, are the peripheral developers, the people who intermittently submit contributions. The core group reviews the patches before they are incorporated into the project. \cite{Crowston2006}

%Describe the context for which our study is focusing on
%Describe the problem we are trying to solve
In this thesis we investigate a contemporary phenomenon (the differences in requirement elicitation between community- and firm-driven OSS projects) in depth and within its real-life context with the boundaries between phenomenon and context not being clearly evident\cite{RobertK1994}.This leads to several challenges. First, software development is a knowledge-intensive activity with a large number of potentially confounding factors \cite{Curtis1986}, and this makes it difficult or impossible to discern the impact of commercial involvement. Second, to observe the impact of commercial involvement, it is important to compare the differences in contributions from developers, and to observe the work impact on projects over long period of time. Third, there is no easy way to learn companies, intentions motivating their involvement in the OSS community, and it is even more difficult to examine the effects of such involvement. 

% related work
\label{related_work}Out of the previous work that has been made on OSS there are two studies that are similar to ours. In the first the authors conduct two case studies from which they try to form theories on what it is that defines OSS-projects. They conclude the study by comparing their results with commercial, proprietary, projects\cite{Mockus2002a}. The second is a case study where the authors investigate the differences in requirement elicitation between large and small OSS-projects\cite{Noll}.
%Describe why we want to solve it (goal)

%Deeper understanding of the process that OSS development 
The purpose of this study is that it will be able to be used as support for firms when considering if they want to make a software open source. Whether it would be an existing software or they are starting from scratch. However we will not research on the comparison between developing OSS and proprietary software. This is important because the results from our study will not list the pros and cons of moving from proprietary software- to OSS-development. Instead they will help the firm realise the potential differences of OSS-projects depending on the governing body.

As a result of this study we will have made the following three contributions. (a) As mentioned above our results can be used as support for a firm when deciding if they want to use OSS development. (b) When studying the characteristics of two OSS-projects the results will of course contribute to the general knowledge about OSS development. (c) The scripts we will develop and use to process the documented discussions(history) of the two projects.

This article presents a comparative case study of differences in requirement elicitation between firm-driven and community-driven OSS projects on Github: the hackable text editor Atom and CodeLite an cross platform IDE. We address key questions about their differences towards each other overall and within the area of requirement elicitation, based on data gathered from Github. Based on the work of \cite{Mockus2002a} and \cite{Noll} we framed a number of hypotheses that we conjectured would be true generally of both development and requirement elicitation within the area of OSS-development. 

%TODO: Shortly describe structure of report
Here we will describe the structure of the thesis and what we will include in each section.
\subsection{Research Question}
%Describe how this study will contribute to a solution or solve it completely
\begin{itemize}
	\item[Q1:]\emph{What are the characteristics of community-driven OSS development?}
	\item[Q2:]\emph{What are the characteristics of firm-driven OSS development?}
	\item[Q3:]\emph{How can software companies benefit from engaging in corporate-driven OSS development?}
\end{itemize}
%-----------------------------------------------------------------------------------------------------------------------------------------------------------------------
\section{Methodology}
%introduce the reader to the section
Here we will introduce the reader to the methodology section...

%Descirbe what we want to research 
The purpose of this study is to compare the effects two different types of governing bodies, a firm or a non-profit community, have on OSS-projects. Given the nature of them both being OSS almost all information will be digital and most of it also publicly documented, including both code evolution and discussions between project participants. With these given conditions we've chosen to conduct a comparative case study on two OSS-projects, one of each type of governing bodies mentioned earlier.

% why not survey
With the given conditions mentioned in the previous paragraph one could argue that conducting a survey would be more suitable. By programmatically processing the information from OSS-projects we could cover a greater amount of projects than if we would do it partially manually. However, such a survey would risk to miss out on important information which could not be parsed.
\subsection{Case Description}
In this section we will describe the two projects we intend to do our case studies on. First we describe Atom, then we describe CodeLite.
\subsubsection{Case 1: Atom}
%TODO

\subsubsection{Case 2: CodeLite}
%TODO

\subsection{Case Study}
%what is a case study?
In \cite{Oates2005} the author refers to a case study as being research strategy for which one tries to describe a case, or phenomenon, from within its natural context. Rather than just confirming a phenomenons existence, which could be done by e.g. conducting a survey, a case study aims to discover why it occurred. This is done, as a researcher, by aiming to give a detailed description of not just the phenomenon but also it's context. It is from the context the researcher can identify factors which might have caused the phenomenon to occur. The amount of factors varies from case to case and the probability of multiple causing factors should be taken into account.

%internet case studies
In \cite{Oates2005} the author says that case study can favourably be used when ''studying the 'life' on the Internet''. However the author mentions particular issues with conducting this kind of research. The first is the problem of boundary. I.e. how do one define the boundaries for which the study is being conducted. The second is the problem of offline/online existence. I.e. one might not get all the details for case by only studying it online, since some information might only be available offline. E.g. communications between participants of the studied case. These problems are taken in to consideration when constructing our method for conducting the case study. However, we believe that the first problem will have minimal effect on our research since it's easy to define what is part and not of the development process of a product and what is not. The second problem on the other hand could affect our research to greater extent. The nature of OSS is that everything related to the project is stored and displayed online. But even though developers might never meet face-to-face, they may still communicate on other channels than those provided through the project. These channels are not always displayed online and can therefore be seen, from a researchers perspective, as offline.

%what have others done in their case studies?
% apache and mozilla
As previously mentioned (section \ref{related_work}) there has been two studies conducted that is similar to ours. In the first, \cite{Mockus2002a}, the authors conduct an exploratory case study with the purpose of mapping how OSS-projects are conducted to form hypothesises on how they generally will perform. The authors also tries to compare their results with commercial projects. To make their generalisation trustworthy they have selected two cases which can be considered typical instances for an OSS-project. The authors present 6 research questions for which they motivate that it should be sufficient to compare with other projects \cite{Mockus2002a}. The 6 questions are presented in schema \ref{apache_mozilla_schema}. In our case study we will use parts of the presented questions. However, the majority of our data collection schema will be similar to the second study.

% firefox and openemr (req. elicitation)
The second study, \cite{Noll}, is also an case study on OSS-projects. But instead of being exploratory it is specific on what factors to compare between two projects, namely requirement elicitation. The authors have chosen to compare requirement elicitation between two extremes off OSS-projects, large and small in terms of end-users of the product and contributors to the project. In order to conduct the study the authors present a method including 5 steps for retrieving the data upon which they will make the comparison. The method is presented in schema \ref{firefox_openemr_schema}.

%what do we want to do in our case study?
In the following sections we present the approach for conducting our case study. We will mainly make use of the method presented in schema \ref{firefox_openemr_schema}. However we will make some modifications to it and add parts of schema \cite{Mockus2002a}.

\subsection{Data Collection}
\subsubsection{Sources}
\begin{itemize}
	\item Github, (issuetracker and commit and merge history)

	\item We need to investigate if the projects have any other channel for communication. If so we should write it here.
\end{itemize}

\subsubsection{Collection process}
\begin{itemize}
	\item This subsection will include: How will we decide what data to collect? what is a requirement? what versions or timespan will we choose? The content right now needs to be edited.
	\item where are the developers geographically from?
\end{itemize}


Previous work on the subject suggests an approach for researching requirement elicitation \cite{Noll}. In the study the authors uses 5 steps for conducting the case study. Their suggested method would however to some extent limit the scope of this study. Therefore we will do some modifications to it. Each step with it's modifications is described below.

\begin{itemize}
	\item[Step 1:] Select a version of the software for which there has been added a set of features.
	\item[Step 2:] Select a subset of these features to examine. We do not consider bugfixes to be a feature, but other than that we think that all features should be selected for examination.
	\item[Step 3a:] Examine resources related to the project, such as archives of discussion forums and the projects issue database to discover how the feature was first proposed.
	\item[Step 3b:] Determine when the feature was acknowledged as a requirement. E.g. a ''core dev'' approving of the suggestion.
	\item[Step 4a:] Determine the initial implementation of the feature submitted.
	\item[Step 4b:] Determine the final implementation of the feature submitted.
	\item[Step 5:] Categorize the requirement as asserted by a developer, either from his or her personal experience or knowledge of user needs; proposed by an end-user, for example by posting a re- quest to one of the discussion forums, or filing a bug report or ?Request for Enhancement? in the issue database; or derived from\footnote{TODO: rephrase step 5}
\end{itemize}
\subsubsection{Data}
What data needs to be collected? Make schema for this (based on similar studies)
\subsubsection{Tools}
\begin{itemize}
	\item Explain Github API

	\item Explain how we will be constructing scripts to gather and analyse data from Github
\end{itemize}
\subsection{Data Analysis}
TODO: write structure for this section
\subsection{Threats to Validity}
\begin{itemize}
	\item When performing a case study it is important to ensure that the context which is being researched is not affected by the researcher.

	\item A generalisation from our results might not be representable for all firm- and community-driven OSS-projects

	\item Github as only source...
\end{itemize}

\subsection{Hypothesis}
It has been shown in OSS-projects that the requirement elicitation process is independent of project size\cite{Noll}. We think that this study will find that the process is also independent of a firm or a community being the governing body.

\section{Result}

\section{Analysis}

\section{Discussion}

\section{Conclusion and Related Work} % TODO Conlusion and Related Work as separate sections?

\newpage
\addcontentsline{toc}{section}{References}
\bibliographystyle{plain}
\bibliography{library}

\end{document}
