\documentclass[a4paper,11pt]{article}

%
% Do not change
\textheight = 220mm
\textwidth  = 150mm
\topmargin  = 10mm
\oddsidemargin  = 5.0mm
\evensidemargin = 5.0mm
\unitlength = 1mm


\usepackage[T1]{fontenc} % Use latin1 font encoding, permits the direct use of ������ in the text

\usepackage[swedish]{babel} % If in Swedish

\begin{document}

\let\rempage=\thepage
{
\renewcommand{\thepage}{\relax}
\begin{picture}(44,0)(15,10)%
\special{psfile=mahlogo-name.eps}
\end{picture}%

\vspace*{-30mm}
\hfill\begin{minipage}[t]{10em}\large
Teknik och samh�lle\\
Datavetenskap
\end{minipage}

\vspace*{45mm}
\begin{center}
{\bf\large
Examensarbete 

\small
15 h�gskolepo�ng, grundniv�
}

\vspace*{25mm}
\LARGE

The differences in requirement elicitation between community- and firm-driven open source software projects on Github.

\vspace*{8mm}
\large

Examensarbetets titel p� svenska %TODO

\vspace*{12mm}
\Large
%
% Author names
Teddy Andersson\\
Filip Harald

\vspace*{30mm}
\large
% picture TODO?
\end{center}

\vfill
\hspace*{-10mm}%
\begin{minipage}[t]{20em}
%
% Fill in correct data for you
Examen:~kandidatexamen 180~hp
\\
Huvudomr�de:~datavetenskap
\\
Program:~systemutvecklare
\\
Datum f�r slutseminarium:~2016-xx-xx %TODO
\end{minipage}
\hfill
\begin{minipage}[t]{15em}
Handledare: Nancy Russo
\\
Examinator: ABC
\end{minipage}

\newpage

\mbox{}

\newpage

\section*{Sammanfattning}

Text p� svenska\ldots

\newpage

\mbox{}

\newpage

\section*{Abstract}

Text in English\ldots

\newpage

\mbox{}

\newpage
\tableofcontents
\newpage
\ifodd\value{page}\else\mbox{}\newpage\fi
\setcounter{page}{1}
\renewcommand{\thepage}{\rempage}

\section{Introduction}

\subsection{Background}
%Brief description of software development

%Introduce the reader to the topic OSS (Open Source Software)
''OSS\footnote{Open Source Software, for the remainder of this article we will refer to this as OSS} [\dots] is characterised by (a) the distribution of a program's source code (programming instructions) along with the binary version of the product and (b) a license that allows a user to make unlimited copies of, and modifications to, this product.'' \cite{DiBonaChrisandOckman1999, Maccormack2006}. OSS is commonly developed by a community that cooperate via the Internet and never, or seldom, meet face to face. The number of developers can differ from a handful to thousands and are often geographically distributed. Developers voluntarily contributes to the software by implementing new features, fixing bugs and writing documentation \cite{Maccormack2006}.
%Describe the context for which our study is focusing on
%Describe the problem we are trying to solve
%Describe why we want to solve it (goal)
%Shortly describe structure of report

\subsection{Research Question}
%Describe how this study will contribute to a solution or solve it completely

\section{Methodolodgy}
\subsection{Method Introduction}
%Describe the ecosystem in which we research. Why it is easy to get historical data.
The domain, OSS-development, in which we are researching is from a researcher perspective a very friendly environment. In that it offers a lot of information to research on. Both the traceability on where in the code work has been done and by whom. But most importantly, by being ''open source'' it's also possible to view the history of the development process. Furthermore given the nature of the information also being digital and structured in a specific pattern it can be processed automatically and hence increasing the amount of information retrieved than if we were to process it manually.
\subsection{Method Description}

\subsection{Metod Discussion}

\section{Result}

\section{Analysis}

\section{Discussion}

\section{Conclusion and Related Work} % TODO Conlusion and Related Work as separate sections?

\newpage
\addcontentsline{toc}{section}{References}
\bibliographystyle{plain}
\bibliography{library}

\end{document}