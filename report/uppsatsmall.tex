\documentclass[a4paper,11pt]{article}

%
% Do not change
\textheight = 220mm
\textwidth  = 150mm
\topmargin  = 10mm
\oddsidemargin  = 5.0mm
\evensidemargin = 5.0mm
\unitlength = 1mm


\usepackage[T1]{fontenc} % Use latin1 font encoding, permits the direct use of ������ in the text

%
% Choose if you write in Swedish or English. (Don't write in English!)
\usepackage[swedish]{babel} % If in Swedish
%\usepackage[english]{babel} % If in English

\begin{document}
%
% Do not change
\let\rempage=\thepage
{
\renewcommand{\thepage}{\relax}
\begin{picture}(44,0)(15,10)%
\special{psfile=mahlogo-name.eps}%
\end{picture}%

\vspace*{-30mm}
\hfill\begin{minipage}[t]{10em}\large
Teknik och samh�lle\\
Datavetenskap
\end{minipage}

\vspace*{45mm}
\begin{center}
{\bf\large
Examensarbete 

\small
15 h�gskolepo�ng, grundniv�
}

\vspace*{25mm}
\LARGE
%
% Put your title here
Examensarbetets titel

\vspace*{8mm}
\large
%
% Translated title 
Examensarbetets titel p� engelska\\
(p� svenska om arbetet �r skrivet p� engelska)

\vspace*{12mm}
\Large
%
% Author names
F�rfattares namn\\
var och en p� egen rad, i bokstavsordning efter efternamn

\vspace*{30mm}
\large
%
% Picture if you want
Eventuell bild
\end{center}

\vfill
\hspace*{-10mm}%
\begin{minipage}[t]{20em}
%
% Fill in correct data for you
Examen:~kandidatexamen 180~hp
\\
Huvudomr�de:~datavetenskap
%
% Huvudomr�den �r:
% Aff�rssystem
% Data och informationsvetenskap (IA och IS)
% Datavetenskap (�vriga)
\\
Program: (t.ex.~systemutvecklare)
\\
Datum f�r slutseminarium: (t.ex.~2012-05-30)
\end{minipage}
%
\hfill
%
\begin{minipage}[t]{15em}
%
% Fill in supervisor and second reader
Handledare: XYZ
\\
Examinator: ABC
\end{minipage}

\newpage

\mbox{}

\newpage

\section*{Sammanfattning}

Text p� svenska\ldots

\newpage

\mbox{}

\newpage

\section*{Abstract}

Text in English\ldots

\newpage

\mbox{}

\newpage
\tableofcontents
\newpage
\ifodd\value{page}\else\mbox{}\newpage\fi
\setcounter{page}{1}
\renewcommand{\thepage}{\rempage}
%


\section{Inledning}

\subsection{Bakgrund och tidigare forskning}

I dagens samh�lle\ldots

\vdots

\ldots\ Lenhart~{\em m.fl.}~\cite{rectlinkfirst} visar en
kvadratisk algoritm f�r problemet. Det f�rb�ttras senare till $O(n\log n)$ tid av
Djidjev~{\em m.fl.}~\cite{rectlinkopt1} och Ke~\cite{rectlinkopt2} oberoende av varandra.~\ldots

Notera att inledningen skrivs i presens, den visar en nul�gessituation.


\subsection{Fr�gest�llning}

\section{Metod}

\subsection{Metodbeskrivning}

\subsection{Metoddiskussion}

\section{Resultat}

\section{Analys}

\section{Diskussion}

\section{Slutsatser och vidare forskning}


%
% Do not change
\newpage
\addcontentsline{toc}{section}{Referenser}
\begin{thebibliography}{888}
%
% Referenserna ordnade i bokstavsordning efter f�rstef�rfattaren

\bibitem{rectlinkopt2}
H.~N.~Djidjev, A.~Lingas, J.-R.~Sack, An $O(n\log n)$ algorithm for
computing the rectilinear link center of a simple polygon. {\em Discrete and
Computational Geometry\/}~8:131--152,~1992.

\bibitem{rectlinkopt1}
Y.~Ke, An efficient algorithm for link distance problems. In {\em
Proceedings of the 5th ACM Symposium on Computational Geometry}, pages~69--78,~1989.

\bibitem{rectlinkfirst}
W.~Lenhart, R.~Pollack, J.-R.~Sack, R.~Seidel, M.~Sharir, S.~Suri,
G.~Toussaint, S.~Whitesides, C.~Yap, Computing the link center of a simple
polygon. {\em Discrete and Computational Geometry\/}~3:281--293,~1988.

\end{thebibliography}

\end{document}