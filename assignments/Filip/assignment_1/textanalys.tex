\documentclass[11pt, oneside]{article}   	% use "amsart" instead of "article" for AMSLaTeX format
\usepackage{geometry}                			% See geometry.pdf to learn the layout options. There are lots.
\geometry{a4paper}

\usepackage[parfill]{parskip}    			% Begin paragraphs with an empty line rather than an indent

\usepackage[swedish]{babel}			% Swedish
\usepackage[T1]{fontenc}				% Swedish
\usepackage[latin1]{inputenc}			% Swedish
%\usepackage{amssymb}

\title{Textanalysuppgift 1}
\author{Filip Harald, ae8556}

\begin{document}
\maketitle
\tableofcontents

\newpage
\section{Requirements Elicitation in Open Source Software
Development: A Case Study}
Artikeln skrevs av John Noll och Wei-Ming Liu \cite{Noll2010}. John Noll publicerade sin f�rsta artikel 1991 \cite{Angeles1991}. Under �ren 2010-2015 har han bl.a. skrivit 4 vetenskapliga artiklar\cite{Noll2010, Monasor2014a, Monasor2014, Clear2015} p� temat ''open source'' och GSD\footnote{Global Software Development}. Wei-Ming Lius tidigare arbeten har p� omr�det skiljt sig fr�n \cite{Noll2010}. De artiklarna har en mer teknisk pr�gel och syftar till att utforma, f�rb�ttra eller testa algoritmer \cite{Li,Yang2011}. 

\cite{Noll2010} publicerades i \emph{''FLOSS-3 workshop\footnote{Proceedings of the 3rd International Workshop on Emerging Trends in Free/Libre/Open Source Software Research and Development}''}. Workshoppen syftade till att analysera konkurrerande ''open source''-projekt f�r att identifiera bl.a.: hur projekten styrs, hur social och teknisk kommunikation sker mellan projekten, hur eventuell korporativ p�verkan sker och hur beroenden och �teranv�ndning sker mellan projekten. 

\newpage
\section{Paper 2 (reviews, user rating etc?)}

\newpage
\section{Paper 3 (open source or user rating?)}

\bibliographystyle{plain}
\bibliography{references}
\end{document}
