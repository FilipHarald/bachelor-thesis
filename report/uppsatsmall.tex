\documentclass[a4paper,11pt]{article}

%
% Do not change
\textheight = 220mm
\textwidth  = 150mm
\topmargin  = 10mm
\oddsidemargin  = 5.0mm
\evensidemargin = 5.0mm
\unitlength = 1mm


\usepackage[T1]{fontenc} % Use latin1 font encoding, permits the direct use of ������ in the text

\usepackage[swedish]{babel} % If in Swedish

\begin{document}

\let\rempage=\thepage
{
\renewcommand{\thepage}{\relax}
\begin{picture}(44,0)(15,10)%
\special{psfile=mahlogo-name.eps}
\end{picture}%

\vspace*{-30mm}
\hfill\begin{minipage}[t]{10em}\large
Teknik och samh�lle\\
Datavetenskap
\end{minipage}

\vspace*{45mm}
\begin{center}
{\bf\large
Examensarbete 

\small
15 h�gskolepo�ng, grundniv�
}

\vspace*{25mm}
\LARGE

The differences in requirement elicitation between community- and firm-driven open source software projects.

\vspace*{8mm}
\large

Examensarbetets titel p� svenska %TODO

\vspace*{12mm}
\Large
%
% Author names
Teddy Andersson\\
Filip Harald

\vspace*{30mm}
\large
% picture TODO?
\end{center}

\vfill
\hspace*{-10mm}%
\begin{minipage}[t]{20em}
%
% Fill in correct data for you
Examen:~kandidatexamen 180~hp
\\
Huvudomr�de:~datavetenskap
\\
Program:~systemutvecklare
\\
Datum f�r slutseminarium:~2016-xx-xx %TODO
\end{minipage}
\hfill
\begin{minipage}[t]{15em}
Handledare: Nancy Russo
\\
Examinator: ABC
\end{minipage}

\newpage

\mbox{}

\newpage

\section*{Sammanfattning}

Text p� svenska\ldots

\newpage

\mbox{}

\newpage

\section*{Abstract}

Text in English\ldots

\newpage

\mbox{}

\newpage
\tableofcontents
\newpage
\ifodd\value{page}\else\mbox{}\newpage\fi
\setcounter{page}{1}
\renewcommand{\thepage}{\rempage}

\section{Introduction}

\subsection{Background}
%Brief description of software development
Since the late 90s, The Open Source Software (OSS\footnote{Open Source Software, for the remainder of this article we will refer to this as OSS}) development model has gained momentum and is now in a commercial settings seen as a viable approach| \cite{Author2008}. During the same period of time Agile models for Information Systems Development (ISD) have been a major game changer in software development. Projects in the Agile models are claimed to encourage developers to be more flexible and efficient by means of arrangements in the development teams physical and social environment.\cite{Jansson2015}. The encouragement from the Agile models on team can likewise have been a participating factor to the momentum of the OSS development model. Looking  deeper into how the work is conducted in the OSS Its interesting to ask how, why and in what context agile method works. Could the momentum of OSS developments model be due to the raise of Agile methods but with more economical benefits from a commercial settings standpoint. Similar to outsourcing, and particular offshore sourcing, the OSS development model promises many advantages. including reduced salary costs; reduced cycle time arising from "follow-the-sun" software development; cross-site modularization of development work; access to a larger skilled developer pool; innovation and shared best practice; and closer proximity to customers. \cite{Author2008}. Explained in \cite{Crowston2006} is that OSS communities, whether run by volunteers or by companies, have a similar structure. The structure have clear lines to how Agile team looks like. At the center of the structure is the core group?this is the group of people/developers who are the key decision makers for the project. Beyond this core group, are the peripheral developers?these are persons who intermittently submit contributions. The core group reviews the patches before they are incorporated into the project. \cite{Crowston2006}


%Introduce the reader to the topic OSS (Open Source Software)
OSS\footnote{Open Source Software, for the remainder of this article we will refer to this as OSS} ''\dots is characterized by (a) the distribution of a program's source code (programming instructions) along with the binary version of the product and (b) a license that allows a user to make unlimited copies of, and modifications to, this product.'' \cite{DiBonaChrisandOckman1999, Maccormack2006}.

%Describe the context for which our study is focusing on
%Describe the problem we are trying to solve
%Describe why we want to solve it (goal)
%Shortly describe structure of report

\subsection{Research Question}
%Describe how this study will contribute to a solution or solve it completely

\section{Method}

\subsection{Method Description}

\subsection{Metod Discussion}

\section{Result}

\section{Analysis}

\section{Discussion}

\section{Conclusion and Related Work} % TODO Conlusion and Related Work as separate sections?

\newpage
\addcontentsline{toc}{section}{References}
\bibliographystyle{plain}
\bibliography{library}

\end{document}