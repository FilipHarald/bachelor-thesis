\documentclass[a4paper,11pt]{article}

%
% Do not change
\textheight = 220mm
\textwidth  = 150mm
\topmargin  = 10mm
\oddsidemargin  = 5.0mm
\evensidemargin = 5.0mm
\unitlength = 1mm


\usepackage[T1]{fontenc} % Use latin1 font encoding, permits the direct use of ������ in the text

\usepackage[swedish]{babel} % If in Swedish

\usepackage{csquotes} % For blockquote

\begin{document}

\let\rempage=\thepage
{
\renewcommand{\thepage}{\relax}
\begin{picture}(44,0)(15,10)%
\special{psfile=mahlogo-name.eps}
\end{picture}%

\vspace*{-30mm}
\hfill\begin{minipage}[t]{10em}\large
Teknik och samh�lle\\
Datavetenskap
\end{minipage}

\vspace*{45mm}
\begin{center}
{\bf\large
Examensarbete 

\small
15 h�gskolepo�ng, grundniv�
}

\vspace*{25mm}
\LARGE

The differences in requirement elicitation between community- and firm-driven open source software projects on Github.

\vspace*{8mm}
\large

Examensarbetets titel p� svenska %TODO

\vspace*{12mm}
\Large
%
% Author names
Teddy Andersson\\
Filip Harald

\vspace*{30mm}
\large
% picture TODO?
\end{center}

\vfill
\hspace*{-10mm}%
\begin{minipage}[t]{20em}
%
% Fill in correct data for you
Examen:~kandidatexamen 180~hp
\\
Huvudomr�de:~datavetenskap
\\
Program:~systemutvecklare
\\
Datum f�r slutseminarium:~2016-xx-xx %TODO
\end{minipage}
\hfill
\begin{minipage}[t]{15em}
Handledare: Nancy Russo
\\
Examinator: ABC
\end{minipage}

\newpage

\mbox{}

\newpage

\section*{Sammanfattning}

Text p� svenska\ldots

\newpage

\mbox{}

\newpage

\section*{Abstract}

Text in English\ldots

\newpage

\mbox{}

\newpage
\tableofcontents
\newpage
\ifodd\value{page}\else\mbox{}\newpage\fi
\setcounter{page}{1}
\renewcommand{\thepage}{\rempage}

\section{Introduction}

\subsection{Background}
%Brief description of software development
Since the late 90s the OSS\footnote{Open Source Software, for the remainder of this article we will refer to this as OSS} development model has gained momentum and is now, in a commercial settings, seen as a viable approach\cite{Author2008}. During the same period of time agile models for information systems development have been a major game changer in software development. Projects in the agile models are claimed to encourage developers to be more flexible and efficient by means of arrangements in the development teams physical and social environment\cite{Jansson2015}. The encouragement from the agile models on team can likewise have been a participating factor to the momentum of the OSS development model. Looking  deeper into how the work is conducted in the OSS development its interesting to ask how, why and in what context agile method works. Could the momentum of OSS developments model be due to the raise of agile methods but with more economical benefits from a commercial settings standpoint. Similar to outsourcing, and particular offshore sourcing, the OSS development model promises many advantages. including reduced salary costs; reduced cycle time arising from''follow-the-sun\footnote{?}'' software development; cross-site modularization of development work; access to a larger skilled developer pool; innovation and shared best practice; and closer proximity to customers\cite{Author2008}. Explained in \cite{Crowston2006} is that OSS communities, whether run by volunteers or by companies, have a similar structure. The structure have clear lines to what an agile team looks like. At the center of the structure is the core group this is the group of people who are the key decision makers for the project. Beyond this core group, are the peripheral developers, the people who intermittently submit contributions. The core group reviews the patches before they are incorporated into the project. \cite{Crowston2006}

%Introduce the reader to the topic OSS (Open Source Software)
\begin{displayquote}
OSS [\dots] is characterised by (a) the distribution of a program's source code (programming instructions) along with the binary version of the product and (b) a license that allows a user to make unlimited copies of, and modifications to, this product. \cite{DiBonaChrisandOckman1999, Maccormack2006}
\end{displayquote}
OSS is commonly developed by a community that cooperate via the Internet and never, or seldom, meet face to face. The number of developers can differ from a handful to thousands and are often geographically distributed. Developers voluntarily contributes to the software by implementing new features, fixing bugs and writing documentation \cite{Maccormack2006}.
%Describe the context for which our study is focusing on
%Describe the problem we are trying to solve

In this article we investigate a contemporary phenomenon (the differences in requirement elicitation between community- and firm-driven open source software projects) in depth and within its real-life context with the boundaries between phenomenon and context not being clearly evident\cite{RobertK1994}.This leads to several challenges. First, software development is a knowledge-intensive activity with a large number of potentially confounding factors \cite{Curtis1986}, and this makes it difficult or impossible to discern the impact of commercial involvement. Second, to observe the impact of commercial involvement, it is important to compare the differences in contributions from developers, and to observe the work impact on projects over long period of time. Third, there is no easy way to learn companies, intentions motivating their involvement in the OSS community, and it is even more difficult to examine the effects of such involvement. 

To address these challenges, we conducted a comparative case study that investigated the parallel evolution of several products. Comparative case studies cover two or more cases in a way that produces more generalisable knowledge about causal questions\cite{Goodrick2014}. In general, it includes a comparison of conditions that pave the way for causal inference. We selected two projects that were developed from the same specification over a similar period (i.e., parallel evolution) but had a varying nature of commercial involvement practices both between the projects and between epochs within a project. This allowed us to compare the differences between projects and the differences between different epochs of a single project

%Describe why we want to solve it (goal)
	%Deeper understanding of the process that OSS development 
	%Argument that can make companies use OSS development more frequently
This article strive to achieve two separate goals within the area of OSS driven development. First, to obtain a deeper understanding of how OSS development projects are structured, used and also how it's percepted from a contributers point of view. Second, to suggest a set of descriptive key points that together can act as a foundation. The foundation itself should be able to act as a guideline when decissions around applying OSS development are being discussed in the upstart of a project. Overall the goal is to achive and present more research in the area of OSS.
%Shortly describe structure of report

\subsection{Research Question}
%Describe how this study will contribute to a solution or solve it completely

\section{Methodolodgy}
\subsection{Method Introduction}
%Describe the ecosystem in which we research. Why it is easy to get historical data.
The domain, OSS-development, in which we are researching is from a researcher perspective a very friendly environment. In that it offers a lot of information to research on. Both the traceability on where in the code work has been done and by whom. But most importantly, by being ''open source'' it's also possible to view the history of the development process. Furthermore given the nature of the information also being digital and structured in a specific pattern it can be processed automatically and hence increasing the amount of information retrieved than if we were to process it manually.
\subsection{Method Description}

\subsection{Metod Discussion}

\section{Result}

\section{Analysis}

\section{Discussion}

\section{Conclusion and Related Work} % TODO Conlusion and Related Work as separate sections?

\newpage
\addcontentsline{toc}{section}{References}
\bibliographystyle{plain}
\bibliography{library}

\end{document}