\documentclass[a4paper,11pt]{article}

%
% Do not change
\textheight = 220mm
\textwidth  = 150mm
\topmargin  = 10mm
\oddsidemargin  = 5.0mm
\evensidemargin = 5.0mm
\unitlength = 1mm


\usepackage[T1]{fontenc} % Use latin1 font encoding, permits the direct use of åäöÅÄÖ in the text

\usepackage[swedish]{babel} % If in Swedish

\usepackage{csquotes} % For blockquote

\begin{document}

\let\rempage=\thepage
{
\renewcommand{\thepage}{\relax}
\begin{picture}(44,0)(15,10)%
\special{psfile=mahlogo-name.eps}
\end{picture}%

\vspace*{-30mm}
\hfill\begin{minipage}[t]{10em}\large
Teknik och samhälle\\
Datavetenskap
\end{minipage}

\vspace*{45mm}
\begin{center}
{\bf\large
Examensarbete 

\small
15 högskolepoäng, grundnivå
}

\vspace*{25mm}
\LARGE

The differences in requirement elicitation between community- and firm-driven open source software projects on Github.

\vspace*{8mm}
\large

Examensarbetets titel på svenska %TODO

\vspace*{12mm}
\Large
%
% Author names
Teddy Andersson\\
Filip Harald

\vspace*{30mm}
\large
% picture TODO?
\end{center}

\vfill
\hspace*{-10mm}%
\begin{minipage}[t]{20em}
%
% Fill in correct data for you
Examen:~kandidatexamen 180~hp
\\
Huvudområde:~datavetenskap
\\
Program:~systemutvecklare
\\
Datum för slutseminarium:~2016-xx-xx %TODO
\end{minipage}
\hfill
\begin{minipage}[t]{15em}
Handledare: Nancy Russo
\\
Examinator: ABC
\end{minipage}

\newpage

\mbox{}

\newpage

\section*{Sammanfattning}

Text på svenska\ldots

\newpage

\mbox{}

\newpage

\section*{Abstract}

Text in English\ldots

\newpage

\mbox{}

\newpage
\tableofcontents
\newpage
\ifodd\value{page}\else\mbox{}\newpage\fi
\setcounter{page}{1}
\renewcommand{\thepage}{\rempage}

\section{Introduction}

\subsection{Background}
%introduce the reader to the section
	%Introduce the reader to OSS
In the area of software development systems must evolve, or they risk to lose market shares to competitors. However, maintaining such a system is extraordinarily difficult, complicated, and time consuming. The tasks of adding new features, adding support for new hardware devices and platforms, system tuning, and defect fixing all become more difficult as a system ages and grows.\cite{M.W.Godfrey2000} Due to the difficulties of maintaining software systems, a new more open approach of development emerge during the late 90's\cite{Wikipedia2011Dec.142000}. Today this development process is referred to as Open-Source Software (OSS) Development. In the essay The Cathedral and the Bazaar by  Eric S. Raymond the progressive open-source development, is described in the following words.

\begin{displayquote}
A great babbling bazaar of differing agendas and approaches out of which a coherent and stable system could seemingly emerge only by a succession of miracles.
\end{displayquote}

\cite{DiBonaChrisandOckman1999, Maccormack2006}, however provides a more understandable description of the OSS development process 
\begin{displayquote}
OSS [\dots] is characterised by (a) the distribution of a program's source code (programming instructions) along with the binary version of the product and (b) a license that allows a user to make unlimited copies of, and modifications to, this product. \cite{DiBonaChrisandOckman1999, Maccormack2006}
\end{displayquote} 
Further more OSS is commonly developed by a community that cooperate via the Internet and never, or seldom, meet face to face. The number of developers can differ from a handful to thousands and are often geographically distributed. Developers voluntarily contributes to the software by implementing new features, fixing bugs and writing documentation \cite{Maccormack2006}.

	%Introduce the reader to Community-driven projects
The Cathedral and the Bazaar written by  Eric S. Raymond in 1997 describes the distinction between two kinds of software development. Raymond first describes a conventional closed-source development. A kind of development method that is, according to Raymond, like the building of a cathedral; central planning, tight organization and one process from start to finish. The second, however, is the progressive open-source development, which is more like \cite{Wikipedia2011Dec.142000}
\begin{displayquote}
A great babbling bazaar of differing agendas and approaches out of which a coherent and stable system could seemingly emerge only by a succession of miracles.
\end{displayquote}
The latter analogy points to the discussion involved in an Open-Source Software (OSS) development process, which takes use of the OSS development model. 
	
	%Introduce the reader to Firm-driven development
	
	%Explain shortly the diferences in requierment elicitaion between Agile and Open-source development
	
Here we will introduce the reader to this section...
%New description approach

(http://www.catb.org/~esr/writings/cathedral-bazaar/cathedral-bazaar/ar01s03.html) The Cathedral and the Bazaar written by  Eric S. Raymond in 1997 describes the distinction between two kinds of software development. Raymond first describes a conventional closed-source development. A kind of development method that is, according to Raymond, like the building of a cathedral; central planning, tight organization and one process from start to finish. The second, however, is the progressive open-source development, which is more like
\begin{displayquote}
A great babbling bazaar of differing agendas and approaches out of which a coherent and stable system could seemingly emerge only by a succession of miracles.
\end{displayquote}
The latter analogy points to the discussion involved in an Open-Source Software (OSS) development process, which takes use of the OSS development model. The composition of the essay was due to the success of Linux, a Unix-like operating system originally written by Linus Torvalds, but subsequently worked on by hundreds of other developers in the late 90's \cite{M.W.Godfrey2000}, since then the OSS development model has gained momentum and is now, in a commercial settings, seen as a viable approach\cite{Author2008}. The frist analogy that Raymond brings up points towards the conventional methods whom have evolved since the late 90's. In this evolution of project process, Agile methods have become popular for information systems development (IDS)\cite{Jansson2015}. Agile methods are claimed to encourage developers to be more flexible and efficient by means of arrangements in the development teams physical and social environment\cite{Jansson2015}. The encouragement from the Agile methods on teams can likewise have been a participating factor to the momentum of the OSS development. Looking  deeper into how the work is conducted in the OSS development its interesting to ask how, why and in what context agile method works. The momentum of OSS development may correlate to the raise of agile methods but with more economical benefits from a commercial settings standpoint. �gerfalk and Fitzgerald \cite{Author2008} coined the word opensourcing which the explains  have characterized offshore sourcing as "outsourcing to a global workforce. They also point to that opensourcing and OSS development in general promises many advantages. including reduced salary costs; reduced cycle time arising from' 'follow-the-sun'' software development; cross-site modularization of development work; access to a larger skilled developer pool; innovation and shared best practice; and closer proximity to customers. 

Explained in \cite{Crowston2006} is that OSS communities, whether run by volunteers or by companies, have a similar structure. The structure have clear lines to what an agile team looks like. At the center of the structure is the core group this is the group of people who are the key decision makers for the project. Beyond this core group, are the peripheral developers, the people who intermittently submit contributions. The core group reviews the patches before they are incorporated into the project. \cite{Crowston2006}

%Introduce the reader to the topic OSS (Open Source Software)
\begin{displayquote}
OSS [\dots] is characterised by (a) the distribution of a program's source code (programming instructions) along with the binary version of the product and (b) a license that allows a user to make unlimited copies of, and modifications to, this product. \cite{DiBonaChrisandOckman1999, Maccormack2006}
\end{displayquote}
OSS is commonly developed by a community that cooperate via the Internet and never, or seldom, meet face to face. The number of developers can differ from a handful to thousands and are often geographically distributed. Developers voluntarily contributes to the software by implementing new features, fixing bugs and writing documentation \cite{Maccormack2006}.
%Describe the context for which our study is focusing on
%Describe the problem we are trying to solve

In this thesis we investigate a contemporary phenomenon (the differences in requirement elicitation between community- and firm-driven OSS projects) in depth and within its real-life context with the boundaries between phenomenon and context not being clearly evident\cite{RobertK1994}.This leads to several challenges. First, software development is a knowledge-intensive activity with a large number of potentially confounding factors \cite{Curtis1986}, and this makes it difficult or impossible to discern the impact of commercial involvement. Second, to observe the impact of commercial involvement, it is important to compare the differences in contributions from developers, and to observe the work impact on projects over long period of time. Third, there is no easy way to learn companies, intentions motivating their involvement in the OSS community, and it is even more difficult to examine the effects of such involvement. 

%Describe why we want to solve it (goal)

%Deeper understanding of the process that OSS development 
%Argument that can make companies use OSS development more frequently
The purpose of this study is that it will be able to be used as support for firms when considering if they want to make a software open source. Whether it would be an existing software or they are starting from scratch. However we will not research on the comparison between developing OSS and proprietary. This is important because the results from our study will not list the pros and cons of moving from proprietary- to OSS-development. Instead they will help the firm realise the differences\footnote{TODO: Should it be potential differences here?} of OSS-projects depending on the governing body.

As a result of this study we will have made the following three contributions. (a) As mentioned above our results can be used as support for a firm when deciding if they want to use OSS development. (b) When studying the characteristics of two OSS-projects the results will of course contribute to the general knowledge about OSS development. (c) The scripts we will develop and use to process the documented discussions(history) of the two projects.

This article presents a comparative case study of differences in requirement elicitation between firm-driven and community-driven OSS projects on Github: the hackable text editor Atom and CodeLite an cross platform IDE. We address key questions about their differences towards each other overall and within the area of requirement elicitation, based on data gathered from Github. Based on the work of \cite{Mockus2002a} and \cite{Noll} we framed a number of hypotheses that we conjectured would be true generally of both development and requirement elicitation within the area of OSS-development. 
%TODO: Shortly describe structure of report

\subsection{Research Question}
%Describe how this study will contribute to a solution or solve it completely
\begin{itemize}
	\item[Q0:]\emph{What is Atom and CodeLite, and how big are they?} 
	
	Since we are only studying two projects, and we don't want to compare OSS-projects on different domains. We want to ensure that they have roughly the same size(lines of code) and product scope.
	\item[Q1:]\emph{What are the characteristics of community-driven open source development?}
	\item[Q2:]\emph{What are the characteristics of firm-driven open source development?}
	\item[Q3:]\emph{How can software companies benefit from engaging in corporate-driven OSS development?}
\end{itemize}
%-----------------------------------------------------------------------------------------------------------------------------------------------------------------------
\section{Methodology}
%introduce the reader to the section
Here we will introduce the reader to this section...

%Descirbe what we want to research 
Describe what we want to research. (comparing firm and corporate driven) and that we want to do a case study on two projects.

\subsection{Case Study}
%Should we describe what a case study is? (shortly)

\subsection{Case Description}
In this section we will describe the two projects we intend to do our case studies on. Firstly we describe Atom, then we describe CodeLite.
\subsubsection{Case 1: Atom}

\subsubsection{Case 2: CodeLite}

\subsection{Case Study}
%what is a case study?

%what do we want to do in our case study?

Even though the information can be processed all requirements will be unique and can be hard to identify solely by processing the documented discussion. Therefore we can not study a large amount of projects by doing for example a survey. By instead comparing only two projects we decrease the risk of missing out on important details. On the other hand, this could also means that our results will not necessarily apply to OSS-projects i general.\footnote{TODO: We need sources and probably need to elaborate on this section...}

To address these challenges, we conducted a comparative case study that investigated the parallel evolution of several products. Comparative case studies cover two or more cases in a way that produces more generalisable knowledge about causal questions\cite{Goodrick2014}. In general, it includes a comparison of conditions that pave the way for causal inference. We selected two projects that were developed from the same specification over a similar period (i.e., parallel evolution) but had a varying nature of commercial involvement practices both between the projects and between epochs within a project. This allowed us to compare the differences between projects and the differences between different epochs of a single project. 


%Describe case study and how we intend to use it
Previous work on the subject suggests an approach for researching requirement elicitation \cite{Noll}. In the study the authors uses 5 steps for conducting the case study. Their suggested method would however to some extent limit the scope of this study. Therefore we will do some modifications to it. Each step with it's modifications are described below.
\footnote{TODO: We need more here, and also elaborate on what a case study should include.}

\begin{itemize}
	\item[Step 1:] Select a version of the software for which there has been added a set of features.
	\item[Step 2:] Select a subset of these features to examine. We do not consider bugfixes to be a feature, but other than that we think that all features should be selected for examination.
	\item[Step 3a:] Examine resources related to the project, such as archives of discussion forums and the projects issue database to discover how the feature was first proposed.
	\item[Step 3b:] Determine when the feature was acknowledged as a requirement. E.g. a ''core dev'' approving of the suggestion.
	\item[Step 4a:] Determine the initial implementation of the feature submitted.
	\item[Step 4b:] Determine the final implementation of the feature submitted.
	\item[Step 5:] Categorize the requirement as asserted by a developer, either from his or her personal experience or knowledge of user needs; proposed by an end-user, for example by posting a re- quest to one of the discussion forums, or filing a bug report or ?Request for Enhancement? in the issue database; or derived from\footnote{TODO: rephrase step 5}
\end{itemize}
\subsection{Data Collection}
\subsubsection{Datasources}
% Github
	% issuetracker
	% commit and merge history
	% forum?
\subsection{Data Analysis}
\subsection{Threats to Validity}
% Github as only source 

\section{Result}

\section{Analysis}

\section{Discussion}

\section{Conclusion and Related Work} % TODO Conlusion and Related Work as separate sections?

\newpage
\addcontentsline{toc}{section}{References}
\bibliographystyle{plain}
\bibliography{library}

\end{document}
