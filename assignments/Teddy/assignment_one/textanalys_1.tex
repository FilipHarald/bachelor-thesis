\documentclass[a4paper]{article}

\usepackage[english]{babel}
\usepackage[utf8]{inputenc}
\usepackage{mathtools}
\usepackage{amssymb}
\usepackage{graphicx}
\usepackage[colorinlistoftodos]{todonotes}
\usepackage{parskip}
\usepackage{fullpage}

\title{Textanalysuppgift 1}
\author{Teddy Andersson, AE5701}

\begin{document}
\maketitle
\tableofcontents

\newpage
\section{A Case Study of Open Source Software Development: The Apache Server}
\subsection{Sammanfattning}
I tidskriftsartikel “A Case Study of Open Source Software Development: The Apache Server” undersöker författarna den anmärkningsvärda utvecklingsprocessen runt OSS(Open Source Software) och hur en större grupp av människor, utan fysisk kontakt, lyckas åstadkomma mjukvaruprodukter med hög kvalite och användning. Artikeln klargör även inledningsvis att det finns fyra huvudsakliga punkter som utgör en skillnad gentemot de mer traditionella tillvägagångssätt för mjukvaruutveckling. Författarna av artikeln klargör att den punkten för denna typ av utvecklingsmetod präglas av volontärer inom mjukvarubranschen. Vidare så beskriver de att arbete i denna typ av utvecklingsprocess inte tilldelas utan personer tar arbete som de är intresserade av. Fortsättningsvis så finns det inte heller någon specificerad design som följs under utvecklingen och avslutningsvis så finns det ingen projektplan, schema eller någon specificerad lista med leverabler. 
För att kunna se hur mjukvaruutveckling bedrivs med en OSS metod så har författarna valt att undersöka utvecklingsprocessen av Apache(referens) i olika aspekter. Detta för att få en bättre förståelse av nyckel variablerna i en OSS baserad utvecklingsprocess. Men samtidigt förklarar de intresset av att se hur communityn av detta slag undviker oberäkneliga förändringar under processen. 
Resultatet av den fallstudie som författarna av artikeln har utfört är att OSS är en oerhört unik utvecklingsprocess. Det framkommer även att hybrider av OSS och kommersiella metoder är fullt möjligt genom att viss funktionalitet av utvecklingen sker kommersiellt samtidigt som andra görs genom OSS. De kan även efter studien dra slutsatsen att de utvecklare som arbetar med OSS ofta har en genuin vilja att framställa en bra produkt då de själv kan ses som en viktig del av produkten och utvecklingen i slutändan. Denna vilja präglas också av att personer vill bidra och känner att de kan bidra på ett meningsfullt sätt. Vidare påpekar de också att kommersiella processer mycket väl skulle kunna fungera på liknande sätt när det kommer till utveckling av mjukvara.


\subsection{Textanalys}
\subsubsection{Innehåll}
Artikeln är skriven av Audris Mockus, Roy T. Fielding och James Herbsleb. Audris Mockus har 83 publikationer med sig i bagaget och ser sig själv som en digital Arkeologi. Roy T.Fielding har varit aktivt involverad i projektet the world wide web (WWW) sedan 1993. Han har mellan 1994 och 2005 totalt publicerat 5 artiklar som involverar Apache och utöver detta även gjort publikationer kring med anknytning till WWW. James Herbsleb har totalt gjort 102 publikationer mellan åren 1992 och 2016. 20 av dessa publikationer involverar temat “open source”. 
Artikeln “A Case Study of Open Source Software Development: The Apache Server” är publicerad i forumet “ ICSE '00 Proceedings of the 22nd international conference on Software engineering”. International Conference of Spectroscopic Ellipsometry (ICSE) 
är en internationell konferens med fokus på det senaste inom utveckling, vetenskap och tillämpningar av spektroskopiska ellipsometri och tillhörande polarisering beroende mättekniker. Artikeln lämpar sig bra i detta omfång då den under tiden tog upp en markant växande open source approach inom området utveckling.
Under tiden som artikeln publicerades år 2000 så hade webbservern Apache vuxit sig stark på marknaden sedan den introducerades 1995. Men dessutom hade Apache utvecklats med hjälp av open source metoder som också under denna tid var något som vuxit. Konceptet open source hade dock florerat sedan 1990 talet men aldrig tagit fart förens Linux och Apache fick fäste i mjukvarubranschen runt tiden som artikeln är publicerad. Artikeln använder sig av ett enkelt akademiskt språk. Vilket gör att den kan förstås av läsare från olika kategorier och därför lämpar sig för personer som är allmänt intresserade av området. Artikeln ger dessutom  inte ett intryck av att vara skriven från med objektivitet. Detta trots att Roy T.Fielding har varit med och utvecklat Apache och därav kunnat yttra sina egna åsikter genom texten, men detta är inget som påverkar artikeln enligt mig.
Författarna har tagit avstamp från 14 tidskriftsartiklar och två internetbaserade källor som innefattar riktlinjer för Apache och en kvantitativ undersökning från netcraft.com, som inte går att nå. Artikeln har en relativt smal ansats då den endast fokuserar på vissa aspekter från ett visst projekt. Valet att begränsa området till ett specifikt projekt är något de medvetet gjort då de har insikt i utvecklingsprocessen av Apache genom Roy T.Fielding. Vilket är ett klokt beslut enligt mig då de enkelt kan få fram data och information kring ett specifikt projekt istället för att ta ett större ansats och övergripande ge sig in i flera projekt.

\subsubsection{Form och struktur}

Artikeln innefattar följande avsnitt: Abstract, Introduktion, Metod och Data källor, Utvecklingsprocessen för Apache, Kvantitativa Resultat och Hypoteser och Kopiering. Abstract, introduktion och Metod avsnittet tar sammanlagt upp ungefär 25\% av innehållet i artikeln. Vidare så tar avsnittet Utvecklingsprocessen för Apache upp ca 15\% medan avsnittet för resultat tar upp 45\% och slutligen är Hypoteser och Kopiering 15\% i omfattning. Strukturen är vid en första anblick rörig men det kompenseras av den otrolig bra skrivna artikeln som har en klar röd tråd från början till slut och återkopplar återkommande till den väldigt förklarande och ingående frågeställningen som är behjälplig för att skapa en bra struktur igenom hela artikeln. Detta i kombination med bra underrubriker i de större avsnitten väger upp för de annorlunda valen av namngivning på rubriker. Artikeln är i generella termer skriven i presens och författarna har gjort ett val att vara opersonliga/passiva i texten vilket känns mer vetenskapligt då de lägger fokus på artikeln och inte dem själva. Något som är anmärkningsvärt är att de refererar till sig själva i tredje person vilket är ett intressant val och tyder än mer på hur opersonliga de faktiskt är i texten. Detta tillsammans gör att delaktigheten och fokuset ligger på att väcka ett intresse kring ämnet och inte minst frågeställningen. 
När det kommer till litteraturreferenser i texten, så har de valt att benämna dessa i slutet av ett stycke. Vilket jag inte har några problem med att återkoppla till då de håller de beskrivande styckena väldigt korta och bra. Författarna har valt att använda sig av referenshanteringssystemet IEEE i artikeln. Avslutningsvis så tycker jag att författarna har gjort stilistiska och typografiska val som är allmänt vedertagna och vilket bidrar till att bättre fokus på texten och dess budskapet. 


\newpage
\section{Linux and Open Source in the Acadmeic Enterprise}
\subsection{Sammanfattning}
I Tidskriftsartikeln “Linux and Open Source in the Acadmeic Enterprise” inleder man skrivandet med förklara att Open Source Software(OSS) har haft en ökad acceptans sedan introduktionen av metoden under 90-talet. Ett bevis på detta är användningen av operativsystemet Linux för webbservrar, där författarna förklarar att denna kombination har blivit något av en norm. Vidare förklaras att deras intresse ligger i att undersöka VCU’s (Virginia Commonwealth University) övergång från proprietära hård och mjuk -varu lösningar till OSS och handels hårdvara. Där författarnas fokus i artikeln är att granska kriterier för urval, metoder för testning, implementation, evaluerings processen och försäljningen av OSS och handels hårdvara till IT-chefer. 

Författarna av artikeln förklara vidare att VCU traditionellt sätt har uppnått den nivå av behov för datoranvändning som användarna har haft. Detta har man åstadkommit med olika varianter av proprietär och modulär mjukvara. Men under de senare åren har har VCU’s tekniska avdelning kämpat för att hålla jämn takt med de ständigt växande forsknings behoven som universitets community har utvecklat. Utifrån VCU’s ståndpunkt vill författarna visa effektiviteten i användningen av  OSS, proprietär hårdvara och operativsystemet Linux för att lösa verkliga problem av olika varianter.

I fallstudien mot VCU lyckades författarna påvisa att användningen av OSS och proprietär hårdvara möjliggjorde att universitet ökade sin effektivt och förmågor. OSS program som Apache Web Server och Cyrus har lett till att universitetet kunnat implementera nödvändig mjukvara mer effektivt, anpassningsbart och samtidigt minska kostnaderna. De kom fram till att OSS bidrar till en stabilare och mer modulär lösning i samband med bättre support och fördelen med öppen källkod i utvecklingen. För VCU’s räkning så bidrog dessa funktioner även till bättre användning av personalen och deras kunskap samtidigt som utgifterna på hårdvara bättre uppfyller de behov som användarna hade.


\subsection{Textanalys}
\subsubsection{Innehåll}
Artikeln är skriven av Mike Davis, Will O’Donovan, John Fritz och Carlisle Childress. Anmärkningsvärt i denna artikel är att alla författare var under tiden anställda/studerande på Virginia Commonwealth University (VCU). Will O’Donovan och Carlisle Childress har endast gjort en publikation på ACM vilken är denna artikeln. John Fritz har bidragit i tre artiklar tidigare som alla är publicerade 1997. Men inga av dessa artiklar har någon relevans inom området av Open source. Mike Davis har tidigare publicerat en artikel kring migration av standardiserad-email tillsammans med John Fritz. 
Artikeln i sig är publicerad i “SIGUCCS '00 Proceedings of the 28th annual ACM SIGUCCS conference on User services: Building the future, s. ” som är en ACM konferens för Universitet och College i delstaten Virginia i USA med inriktning mot dator tjänster. Det känns som ett lämpligt forum att publicera artikeln i då artikeln i sig faktiskt inriktar sig mot VCU och är skriven av personer som arbetar/studerar på ett universitet i Virginia. Publiceringen av artikeln är intressant ur aspekten att Open Source rörelsen har börjat trenda inom utvecklingsprocesser och vetenskapliga artiklar överlag. Artikeln använder sig av ett enkelt akademiskt språk. Vilket gör att den kan förstås av läsare från olika kategorier men den är skriven mot en mer akademisk kultur och inte minst mot VCU. Som läsare får jag en känsla av att författarna vill väcka en diskussion kring användning av open source mjukvara på universitetet, vilket i så fall än mer pekar på en vinkling mot en akademisk publik. 
Författarna lyckas inte fullt ut hålla en genomgående objektivitet utan går in på ett personligt plan i mindre stycken där en del åsikter lyser igenom, men det är inget som är anmärkningsvärt i första anblick.
Artikeln har totalt 3 referenser som täcker området som de berör. Men den kunde varit betydligt längre och mer komplett, jag anser att de går in på områden i texten där det med all säkerhet finns forskningsunderlag att finna. Exempelvis går författarna in på historisk fakta kring operativsystemet Linux utan att ha någon specifik referens till de påstående som de tar upp. Liknande beetende i akrikeln återkommer och därför känns trovärdigheten inte lika godtagen även om texten i sig är intressant. Avsatsen är smal och fokuserar på Open Source inom VCU. Vilket jag anser vara en ett korrekt då det är en allmänt känd miljö för författarna att forska inom.


\subsubsection{Form och struktur}
Artikeln innefattar följande avsnitt: Abstract, Nyckelord, Introduktion, Hårdvara, Mjukvara, Fallstudier, Slutsatser och Referenser. Abstract, Nyckelord och Introduktionen står för 10\% av artikelns innehåll. Vidare så står avsnitten Hårdvara tillsammans med Mjukvara 20\% av innehållet. Avsnittet som innefattar fallstudier står för 60\% av allt innehåll i artikeln. Avlsutningsvis så står Slutsaser och Refernser för de resterande 10\%. Artikeln från går normerna för hur en vetenskaplig artikel bör vara strukturerade. De har aktivt valt bort en struktur med tydliga rubriker för beskrivning av metod och avgränsning i resultat. Detta val göra att artikeln blir rörig och svår att navigera i, för mig är detta ett väldigt underligt val och jag kan inte sätta fingret på varför de har valt denna typ av struktur på artikeln. Utöver detta så är de underrubriker som finns i artikeln väldigt otydliga och svåra att se i första anblick, vilket leder till en försvårad navigering i artikeln. Det är även väldigt svårt att börja läsa mitt i artikeln om man inte har rätt förkunskaper då författarna har valt att successivt valt att beskriva koncept och termer som tas upp längre fram i texten.
Ett plus som är värt att nämna är att artikeln genomgående har en väldigt lätt stil vilket gör den lättläst. Ambitionen med att skriva den med en lätt stil är förmodligen för att dela den med personer som är insatta i området men även med dem som befinner sig på VCU och inte är insatta i det aktuella området. Det är förmodligen också av denna anledning som Hårdvara och Mjukvara redogörs väldigt grundligt inledningsvis i artikeln. 
Artikeln blandar presens och preteritum återkommande i texten på ett bra sätt och det är inget som direkt påverkar läsandet i artikeln. Däremot så använder författarna vi form i vissa avseende vilket också återspeglar en förlust av objektivitet då man som läsare kan uppfatta vissa delar som personliga för författarna. 
Hur författarna har missat att inkludera referenser i texten är oförståeligt och oerhört illa, inte minst för trovärdigheten av artikeln. Detta lämnar en tomhet i artikeln då du inte kan gå vidare och se vart de har grundat specifika stycken av information. Ännu mer underligt är att de har inkluderat referenser i slutet av artikeln, detta väcker givetvis ytterligare frågetecken kring artikeln och författarna själva. Eftersom det inte finns någon hänvisning till referenserna i texten så är det svårt att avgör vilken typ av referenssystem som används i artikeln. Men baserat på referenslistan så ser det ut som IEE. När det kommer till typografiska och stilistiska val så har författarna använt en tvåspaltig datavetenskaplig struktur på sin artikeln som är väldigt familjär för de som tidigare läst data vetenskapliga artiklar och vetenskapliga artiklar överlag.



\newpage
\section{An Analysis of Requirements Evolution in Open Source Projects: Recommendations for issue Trackers}
\subsection{Sammanfattning}
Artiklen “An Analysis of Requirements Evolution in Open Source Projects: Recommendations for issue Trackers” har i syfte att beskriva en explorativ undersökning kring hur krav hanteras i Open Source Software (OSS) projekt. Författarna av artikeln valde att undersöka detta då de upplevde olikheter i kravhantering för OSS och traditionella projekt, där traditionella projekten kan ta lärdom från hur kravhantering sköts i OSS projekt. Författarnas övergripande idé med artikeln är komma fram till rekommendationer för verktyg och metoder som kan användas för att tackla svårigheter när det kommer till att entra endast giltig funktionalitet i ett system, då detta är ett vanligt problem i OSS projekt där mängden människor är stor och ingen fysisk kommunikation sker. Författarna har förhoppningar att dessa rekommendationer även kan appliceras i mer traditionella projekt. Artikeln beskriver att författarna med hjälp av en kravhanterings programmet Bugzilla så kunde de observera att att det fanns en hög grad av multipla funktions förfrågningar för ett krav. Vidare så tillämpades en lösning i form av klassificering för de multipla förfrågningarna. Författarna förklarar att de genom klassificeringen kunnat ge stöd till verktyg i form av rekommendationer och implikationer för att undvika multipla förfrågningar av ett funktionellt krav. Författarna menar även på att deras huvudmål för framtida arbete är att ytterligare öka stödet för klassificeringen till flera verktyg. De menar även att nästa steg i denna riktningen är att kunna ge användaren av “issue trackers” en bra överblick av projektet i sig. Där det även hade varit en bra idé att kunna se relationer till andra funktionella krav och hur långt gånga dessa är samt vilka som arbetar med dem. Samtidigt så ser de också det som en bra tillämpning att forma bättre hjälpmedel för sökning av krav, för att ytterligare minska antalet multipelt förfrågningar för ett och samma krav. 

\subsection{Textanalys}
\subsubsection{Innehåll}
Artikeln är författad av Petra Heck och Andy Zaidman. Petra Heck har med denna publikationen inräknad medverkat i 4 artiklar som inriktar sig på open source. Dessutom så är all publikationer som hon har gjort på ACM en inriktning på krav och kravhantering. Andy Zaidman har medverkar i 20 publikationer kring open source men har i stor utsträckning skrivit mer åt det tekniska hållet i de andra artiklar som han medverkar. Han har totalt medverkat i 71 publikationer. Artikeln är publicerat i “IWPSE 2013 Proceedings of the 2013 International Workshop on Principles of Software Evolution, pages 43-52”. Eftersom artikeln behandlar just problematik som kan uppstå under evolutionen av ett system så passar den utmärkt in i den tidigare nämnda workshopen. Det nämns inte något i artikeln som motiverar att artikeln skrevs vid just den tidpunkten.
Artikeln är skriven med en väldigt lättsam ton där texten riktar sig mot den intresserade allmänheten. Delvis viktiga begrepp introduceras inte alltid för läsaren och kan därför skapa en viss problematik i läsandet om man inte har en viss förkunskap. Artikeln i helhet ger känslan av rekommendation kring hur metoder och verktyg kan användas för att undvika de problem som tas upp. Av denna anledning så är artikeln då inte objektivt skriven.
Samtliga referenser som artikeln referera till är vetenskapliga artiklar där 3 av totalt 29 källor är författarnas egna publikationer. Källorna är korrekt använda för att understryka påstående och fakta som författarna tar upp i texten. 
Artikeln har en bred ansats, där författarna har som mål att ge rekommendationer. Rekommendationerna är olika tillvägagångssätt för att lösa problematik utifrån ett visst antal frågor. Artikeln ger ingen implicit röd tråd och blir därför svår att följa även om det inte fattas innehåll i texten. Men trovärdigheten blir bristande och arbetet känns inte optimalt framställt.

\subsubsection{Form och struktur}
Artikeln är indelad i följande avsnitt: abstract, introduction, open source requirements, duplicate feature requests, assisting user to avoid duplicate requests, related work, discussion and future work och conclusion. Av de 7 avsnitten så kan endast abstract och introduction läsas utan förkunskap inom området. Artikeln är skriven i tempus och gör läsbarheten mycket enklare trots avsaknaden av en röd tråd genom arbetet. Genom hela artikeln så använder författarna vi-form på ett sätt som involverar läsaren i texten och återigen främjar läsbarheten. 
Gällande referenser så har författarna av artikeln valt att använda oxford som referenssystem. Påstående understryks direkt efteråt och inte i slutet av ett stycke. 
Artikeln har använt sig av standardformatering vilket ökar läsbarheten. Ett bra val då det följer normerna för stilistiska och typografiska val i allmänhet.


\newpage
\bibliographystyle{plain}
\bibliography{references}
\end{document}