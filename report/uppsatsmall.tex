\documentclass[a4paper,11pt]{article}

%
% Do not change
\textheight = 220mm
\textwidth  = 150mm
\topmargin  = 10mm
\oddsidemargin  = 5.0mm
\evensidemargin = 5.0mm
\unitlength = 1mm


\usepackage[T1]{fontenc} % Use latin1 font encoding, permits the direct use of åäöÅÄÖ in the text

\usepackage[swedish]{babel} % If in Swedish

\usepackage{csquotes} % For blockquote

\begin{document}

\let\rempage=\thepage
{
\renewcommand{\thepage}{\relax}
\begin{picture}(44,0)(15,10)%
\special{psfile=mahlogo-name.eps}
\end{picture}%

\vspace*{-30mm}
\hfill\begin{minipage}[t]{10em}\large
Teknik och samhälle\\
Datavetenskap
\end{minipage}

\vspace*{45mm}
\begin{center}
{\bf\large
Examensarbete 

\small
15 högskolepoäng, grundnivå
}

\vspace*{25mm}
\LARGE

The differences in requirement elicitation between community- and firm-driven open source software projects on Github.

\vspace*{8mm}
\large

Examensarbetets titel på svenska %TODO

\vspace*{12mm}
\Large
%
% Author names
Teddy Andersson\\
Filip Harald

\vspace*{30mm}
\large
% picture TODO?
\end{center}

\vfill
\hspace*{-10mm}%
\begin{minipage}[t]{20em}
%
% Fill in correct data for you
Examen:~kandidatexamen 180~hp
\\
Huvudområde:~datavetenskap
\\
Program:~systemutvecklare
\\
Datum för slutseminarium:~2016-xx-xx %TODO
\end{minipage}
\hfill
\begin{minipage}[t]{15em}
Handledare: Nancy Russo
\\
Examinator: ABC
\end{minipage}

\newpage

\mbox{}

\newpage

\section*{Sammanfattning}

Text på svenska\ldots

\newpage

\mbox{}

\newpage

\section*{Abstract}

Text in English\ldots

\newpage

\mbox{}

\newpage
\tableofcontents
\newpage
\ifodd\value{page}\else\mbox{}\newpage\fi
\setcounter{page}{1}
\renewcommand{\thepage}{\rempage}

\section{Introduction}

\subsection{Background}
%New description approach
(http://www.catb.org/~esr/writings/cathedral-bazaar/cathedral-bazaar/ar01s03.html) The Cathedral and the Bazaar written by  Eric S. Raymond in 1997 describes the distinction between two kinds of software development. Raymond first describes a conventional closed-source development. A kind of development method that is, according to Raymond, like the building of a cathedral; central planning, tight organization and one process from start to finish. The second, however, is the progressive open-source development, which is more like
\begin{displayquote}
A great babbling bazaar of differing agendas and approaches out of which a coherent and stable system could seemingly emerge only by a succession of miracles.
\end{displayquote}
The latter analogy points to the discussion involved in an Open-Source Software (OSS) development process, which takes use of the OSS development model. The composition of the essay was due to the success of Linux, a Unix-like operating system originally written by Linus Torvalds, but subsequently worked on by hundreds of other developers in the late 90's \cite{M.W.Godfrey2000}, since then the OSS development model has gained momentum and is now, in a commercial settings, seen as a viable approach\cite{Author2008}. The frist analogy that Raymond brings up points towards the conventional methods whom have evolved since the late 90's. In this evolution of project process, Agile methods have become popular for information systems development (IDS)\cite{Jansson2015}. Agile methods are claimed to encourage developers to be more flexible and efficient by means of arrangements in the development teams physical and social environment\cite{Jansson2015}. The encouragement from the Agile methods on teams can likewise have been a participating factor to the momentum of the OSS development. Looking  deeper into how the work is conducted in the OSS development its interesting to ask how, why and in what context agile method works. The momentum of OSS development may correlate to the raise of agile methods but with more economical benefits from a commercial settings standpoint. �gerfalk and Fitzgerald \cite{Author2008} coined the word opensourcing which the explains  have characterized offshore sourcing as "outsourcing to a global workforce. They also point to that opensourcing and OSS development in general promises many advantages. including reduced salary costs; reduced cycle time arising from' 'follow-the-sun'' software development; cross-site modularization of development work; access to a larger skilled developer pool; innovation and shared best practice; and closer proximity to customers. Explained in \cite{Crowston2006} is that OSS communities, whether run by volunteers or by companies, have a similar structure. The structure have clear lines to what an agile team looks like. At the center of the structure is the core group this is the group of people who are the key decision makers for the project. Beyond this core group, are the peripheral developers, the people who intermittently submit contributions. The core group reviews the patches before they are incorporated into the project. \cite{Crowston2006}

%Introduce the reader to the topic OSS (Open Source Software)
\begin{displayquote}
OSS [\dots] is characterised by (a) the distribution of a program's source code (programming instructions) along with the binary version of the product and (b) a license that allows a user to make unlimited copies of, and modifications to, this product. \cite{DiBonaChrisandOckman1999, Maccormack2006}
\end{displayquote}
OSS is commonly developed by a community that cooperate via the Internet and never, or seldom, meet face to face. The number of developers can differ from a handful to thousands and are often geographically distributed. Developers voluntarily contributes to the software by implementing new features, fixing bugs and writing documentation \cite{Maccormack2006}.
%Describe the context for which our study is focusing on
%Describe the problem we are trying to solve

In this thesis we investigate a contemporary phenomenon (the differences in requirement elicitation between community- and firm-driven open source software projects) in depth and within its real-life context with the boundaries between phenomenon and context not being clearly evident\cite{RobertK1994}.This leads to several challenges. First, software development is a knowledge-intensive activity with a large number of potentially confounding factors \cite{Curtis1986}, and this makes it difficult or impossible to discern the impact of commercial involvement. Second, to observe the impact of commercial involvement, it is important to compare the differences in contributions from developers, and to observe the work impact on projects over long period of time. Third, there is no easy way to learn companies, intentions motivating their involvement in the OSS community, and it is even more difficult to examine the effects of such involvement. 

%Describe why we want to solve it (goal)

%Deeper understanding of the process that OSS development 
%Argument that can make companies use OSS development more frequently
The purpose of this study is that it will be able to be used as support for firms when considering if they want to make a software open source. Whether it would be an existing software or they are starting from scratch. However we will not research on the comparison between developing open source and proprietary. This is important because the results from our study will not list the pros and cons of moving from proprietary- to open source-development. Instead they will help the firm realise the differences\footnote{TODO: Should it be potential differences here?} of OSS-projects depending on the governing body.

As a result of this study we will have made the following three contributions. (a) As mentioned above our results can be used as support for a firm when deciding if they want to use open source development. (b) When studying the characteristics of two OSS-projects the results will of course contribute to the general knowledge about open source development. (c) The scripts we will use to process the documented discussions(history) of the two projects.

This article presents a comparative case study of differences in requirement elicitation between firm-driven and community-driven OSS projects on Github: the hackable text editor ATOM and CodeLite an cross platform IDE. We address key questions about their differences towards each other overall and within the area of requirement elicitation, based on data gathered from Github. Based on the work of \cite{Mockus2002a} and \cite{Noll} we framed a number of hypotheses that we conjectured would be true generally of both development and requirement elicitation within the area of OSS-development. 
%Shortly describe structure of report

%TODO: Shortly describe structure of report

\subsection{Research Question}
%Describe how this study will contribute to a solution or solve it completely
\begin{itemize}
	\item[Q0:]\emph{What is ATOM and CodeLite, and how big are they?} 
	
	Since we are only studying two projects, and we don't want to compare OSS-projects on different domains. We want to ensure that they have roughly the same size(lines of code) and product scope.
	\item[Q1:]\emph{What are the characteristics of community-driven open source development?}
	\item[Q2:]\emph{What are the characteristics of corporate-driven open source development?}
	\item[Q3:]\emph{How can software companies benefit from engaging in corporate-driven OSS development?}
\end{itemize}

\section{Methodology}
\subsection{Introduction}
%Describe the ecosystem in which we research. Why it is easy to get historical data.
The domain, OSS-development, in which we are researching is from a researcher perspective a very friendly environment. In that it offers a lot of information to research on. Both the traceability on where in the code work has been done and by whom. But most importantly, by being ''open source'' it's also possible to observe the history of the development process since most of the communication is done through open channels. Furthermore given the nature of the information also being digital and structured in a specific pattern it can be processed automatically and hence increasing the amount of information we're able to process, than if we were to process it manually.

\subsection{Description}

%Describe case study and how we intend to use it

%Describe the way we have chosen to examine and evaluate the RQ

	%Are there significant differences between the projects
	
	%The difference in requirment elicitation 
		%Define releases in both projects that are equal to one an other in some aspects.
		%Define the process of finding the 

Even though the information can be processed the requirements will be unique and can therefore be hard to identify solely by processing the history. Therefore we can not study a large amount of projects by doing for example a survey. By comparing only two projects we decrease the risk of missing out on important details. On the other hand, this could also means that our results will not necessarily apply to OSS-projects i general.\footnote{TODO: We need sources and probably need to elaborate on this section...}
%Instead we will do two case study on two different projects to see if there are any differences and from that be able to draw generalised conclusion.

So to address these challenges, we conducted a comparative case study that investigated the parallel evolution of several products. Comparative case studies cover two or more cases in a way that produces more generalisable knowledge about causal questions\cite{Goodrick2014}. In general, it includes a comparison of conditions that pave the way for causal inference. We selected two projects that were developed from the same specification over a similar period (i.e., parallel evolution) but had a varying nature of commercial involvement practices both between the projects and between epochs within a project. This allowed us to compare the differences between projects and the differences between different epochs of a single project. 


\subsection{Case Study}
Previous work on the subject suggests an approach for researching requirement elicitation \cite{Noll}. In the study the authors uses 5 steps for conducting the case study. The suggested method will however to some extent limit the scope of the study. And therefore we will do some modifications to it. Each step and it's modifications are described below.

\begin{itemize}
	\item[Step 1:] Select a version of the software for which there has been added a set of features.
	\item[Step 2:] Select a subset of these features to examine. We do not consider bugfixes to be a feature, but other than that we think that all features should be selected for examination.
	\item[Step 3:] Examine resources related to the project, such as archives of discussion forums and the projects issue database to discover when the feature was first proposed, and what role the person proposing the feature played.
\end{itemize}


\subsection{Discussion}

\section{Result}

\section{Analysis}

\section{Discussion}

\section{Conclusion and Related Work} % TODO Conlusion and Related Work as separate sections?

\newpage
\addcontentsline{toc}{section}{References}
\bibliographystyle{plain}
\bibliography{library}

\end{document}
